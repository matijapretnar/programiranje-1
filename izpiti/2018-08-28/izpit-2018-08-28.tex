\documentclass[arhiv]{../izpit}
\usepackage{fouriernc}
\usepackage{xcolor}
\usepackage{tikz}
\usepackage{fancyvrb}
\VerbatimFootnotes{}

\begin{document}

\izpit{Programiranje I: 3.\ izpit}{28.\ avgust 2018}{
  Čas reševanja je 150 minut.
  Veliko uspeha!
}

%%%%%%%%%%%%%%%%%%%%%%%%%%%%%%%%%%%%%%%%%%%%%%%%%%%%%%%%%%%%%%%%%%%%%%%
\naloga[]

\podnaloga
Napišite funkcijo
\begin{verbatim}
    razlika_kvadratov : int -> int -> int,
\end{verbatim}
ki za dani števili izračuna razliko med kvadratom vsote in vsoto kvadratov.

\podnaloga
Napišite funkcijo
\begin{verbatim}
    uporabi_na_paru : ('a -> 'b) -> 'a * 'a -> 'b * 'b,
\end{verbatim}
ki podano funkcijo uporabi na vsakemu od elementov para.

\podnaloga
Napišite funkcijo
\begin{verbatim}
    ponovi_seznam : int -> 'a list -> 'a list
\end{verbatim}
za katero \verb|ponovi_seznam n sez| vrne seznam sestavljen iz \verb|n| ponovitev podanega seznama.
Če je \verb|n| manjši ali enak $0$, naj funkcija vrne prazen seznam.

\podnaloga
Napišite funkcijo
\begin{verbatim}
    razdeli : 'a -> 'a list -> 'a list * 'a list
\end{verbatim}
za katero \verb|razdeli n sez| vrne par seznamov, kjer levi seznam v paru vsebuje vse elemente \verb|sez|,
ki so manjši od \verb|n|, desni pa preostale.
Za vse točke naj bo funkcija \emph{repno-rekurzivna}.

%%%%%%%%%%%%%%%%%%%%%%%%%%%%%%%%%%%%%%%%%%%%%%%%%%%%%%%%%%%%%%%%%%%%%%%
\naloga[]
Konstruiramo podatkovni tip verige filtrov:
\begin{verbatim}
  type 'a filter = | Filter of ('a -> bool) * 'a list * 'a filter
                   | Ostalo of 'a list,
\end{verbatim}
kjer \verb|Filter(f, sez, filtri)| predstavlja člen v verigi, ki v seznamu \verb|sez| hrani elemente,
za katere funkcija \verb|f| vrne \verb|true|, parameter \verb|filtri| pa hrani naslednji člen v verigi.

\podnaloga
Napišite prazen filter (torej takšen, ki še ne hrani nobenega elementa) \verb|test : int filter|,
ki v prvem členu hrani negativna števila, v drugem členu hrani števila manjša od 10 in
v tretjem členu preostala števila.

\podnaloga
Napišite funkcijo \verb|vstavi : 'a -> 'a filter -> 'a filter|,
ki podani element vstavi na začetek seznama tistega člena filtra, ki prvi zadošča filtracijski funkciji.
Pri vstavljanju števil v \verb|test| bi se število $-5$ vstavilo v seznam prvega člena verige,
števili $1$ in $7$ v seznam drugega člena in število $100$ v seznam zadnjega člena.

\podnaloga
Napišite funkcijo \verb|poisci : 'a -> 'a filter -> bool|,
ki preveri, ali je element vsebovan v kateremu od seznamov v verigi filtrov.

\podnaloga
Napišite funkcijo \verb|izprazni_filtre : 'a filter -> 'a filter * 'a list|,
ki kot par vrne verigo filtrov, kjer imajo vsi členi prazne sezname, in seznam vseh
elementov, ki so bili hranjeni v filtrih. Če prvi člen hrani seznam \verb|[1; 2]|
in drugi člen seznam \verb|[3; 4]|, naj bo seznam vseh elementov enak \verb|[1; 2; 3; 4]|.

\podnaloga
Napišite funkcijo \verb|dodaj_filter : ('a -> bool) -> 'a filter -> 'a filter|,
ki na začetek podane verige filtrov doda nov filter s filtracijsko funkcijo \verb|f|.
Elementi v vrnjeni verigi filtrov naj bodo pravilno razporejeni tudi glede na novo dodani člen.

%%%%%%%%%%%%%%%%%%%%%%%%%%%%%%%%%%%%%%%%%%%%%%%%%%%%%%%%%%%%%%%%%%%%%%%
\naloga[]
Študenti tekmujejo v razvoju avtopilota za robotke. Na plošči, kjer je vsako polje ovrednoteno,
je cilj, da robotek nabere čim več točk in konča na ciljnem polju. Ker imajo na voljo
zgolj cenene robotke, avtopilot nima nadzora nad hitrostjo, torej se bo robotek premikal
tako dolgo, kot mu dopušča baterija.

\podnaloga
Polje predstavimo s seznamom seznamov. Napišite testni primer, ki predstavlja ploščo
\[
\begin{bmatrix}
    2 & 4 & 1 & 1 \\
    3 & 2 & 0 & 5 \\
    8 & 0 & 7 & 2
\end{bmatrix}\,.
\]

\podnaloga
Napišite funkcijo \verb|mozni_premiki|, ki za podani koordinati vrne seznam vseh
koordinat, na katere se lahko robotek premakne v naslednjem premiku.

\podnaloga
Napišite funkcijo \verb|ovrednoti|, ki ovrednoti najboljšo možno pot robotka.
Funkcija naj sprejme koordinati začetnega polja, koordinati končnega polja, število premikov
in ploščo z vrednostmi.
Izračuna naj, koliko največ točk lahko robotek nabere na svoji poti. Pri tem lahko
isto polje obišče večkrat, končati pa mora nujno na ciljnem polju. Ker mora pri tem
nujno opraviti vse premike, se lahko zgodi, da ciljnega polja ne more doseči.
V tem primeru naj funkcija vrne \verb|None|.
Za vse točke naj bo funkcija časovno učinkovita.

\begin{verbatim}
    # Primer:
    >>> print(ovrednoti((0,0),(2,2),10,test))
    53
    >>> print(ovrednoti((0,0),(2,2),11,test))
    None
\end{verbatim}

\end{document}
