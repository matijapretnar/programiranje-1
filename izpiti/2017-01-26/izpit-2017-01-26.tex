\documentclass[arhiv]{../izpit}
\usepackage{fouriernc}
\usepackage{xcolor}
\usepackage{tikz}

\begin{document}

\izpit{Programiranje I: 1.\ izpit}{26.\ januar 2017}{
  Čas reševanja je 150 minut.
  \textbf{Funkcij v Haskellu ne pozabite opremiti z ustrezno signaturo.}
  Veliko uspeha!
}

%%%%%%%%%%%%%%%%%%%%%%%%%%%%%%%%%%%%%%%%%%%%%%%%%%%%%%%%%%%%%%%%%%%%%%
\naloga[Aritmetični izrazi, 30 točk]

Aritmetični izraz, ki vključuje le cela števila ter njihove vsote in produkte,
lahko predstavimo s podatkovnim tipom \texttt{Izraz}, ki ga definiramo takole:
\begin{verbatim}
  data Izraz = Samo Int | Plus Izraz Izraz | Krat Izraz Izraz deriving Show
\end{verbatim}
Aritmetičnemu izrazu \texttt{(1 + 2) * 3} tako ustreza izraz \texttt{Krat (Plus (Samo 1) (Samo 2)) (Samo 3)}.

\podnaloga
  Sestavite funkcijo \texttt{izračunaj}, ki izraz evalvira. Na primer:
  \begin{verbatim}
  ghci> izracunaj (Samo 1)
  1
  ghci> izracunaj (Krat (Plus (Samo 1) (Samo 2)) (Samo 3))
  9
  \end{verbatim}

\podnaloga
  Podatkovni tip \texttt{Izraz} napravite za primerek razreda tipov \texttt{Num}.

\podnaloga
  Definirajte podatkovni tip \texttt{RazsirjeniIzraz}, ki ima konstruktorja
  za seštevanje in množenje, ki sprejmeta seznam objektov tipa \texttt{Izraz}.
  ???

%%%%%%%%%%%%%%%%%%%%%%%%%%%%%%%%%%%%%%%%%%%%%%%%%%%%%%%%%%%%%%%%%%%%%%
\naloga[Orbite, 20 točk]

\podnaloga
  Sestavite funkcijo \texttt{orbita}, ki izračuna seznam vseh različnih
  elementov, ki jih lahko dobimo z zaporedno uporabo dane funkcije na danem
  elementu. Na primer:
  \begin{verbatim}
  ghci> orbita (\x -> mod (x + 2) 10) 13
  [13,5,7,9,1,3]
  ghci> orbita succ 5
  [5,6,7,8,9,10,11,...]
  ghci> orbita negate 0
  [0]
  ghci> orbita negate 1
  [1,-1]
  \end{verbatim}

\podnaloga
  Sestavite funkcijo \texttt{generatorji}, ki izračuna najkrajši seznam vseh
  elementov, ki jih potrebujemo, da z zaporedno uporabo dane funkcije dobimo
  vse elemente danega seznama. Na primer:
  \begin{verbatim}
  ghci> generatorji negate [1,-2,-1]
  [1,-2]
  ghci> generatorji (\x -> (x + 3) `mod` 10) [1,2,3,4]
  [1]
  ghci> generatorji (\x -> (x + 2) `mod` 10) [1,2,3,4]
  [1,2]
  ghci> generatorji (\x -> (x + 2) `mod` 10) [1,2,3,4,5,11]
  [2,11] 
  \end{verbatim}
  Če je najkrajših seznamov več, lahko funkcija vrne katerega koli.


%%%%%%%%%%%%%%%%%%%%%%%%%%%%%%%%%%%%%%%%%%%%%%%%%%%%%%%%%%%%%%%%%%%%%%
\naloga[Fiksne točke, 10 točk]

  Sestavite funkcijo \texttt{fiksna}, ki sprejme urejen seznam celih števil 
  $[x_i]_{i=1}^n$ in vrne najmanjši indeks $i$, za katerega velja $x_i = i$.
  Če tak indeks ne obstaja, naj funkcija vrne {\sl None} oziroma {\sl Nothing}.
  Na primer:

  \begin{verbatim}
  ghci> fiksna [-4, -2, 0, 2, 4]
  Just 4
  ghci> fiksna [-4, -2, 0, 3, 4]
  Just 3
  ghci> fiksna [-4, -2, 0, 2, 3]
  Nothing

  >>> fiksna([-4, -2, 0, 2, 4])
  4
  >>> fiksna([-4, -2, 0, 3, 4])
  3
  >>> fiksna([-4, -2, 0, 2, 3])
  None
  \end{verbatim}   

%%%%%%%%%%%%%%%%%%%%%%%%%%%%%%%%%%%%%%%%%%%%%%%%%%%%%%%%%%%%%%%%%%%%%%
\naloga[Nim, 20 točk]

  Igralca $A$ in $B$ igrata igro {\em Vžigalice}. Igralca igrata
  izmenično, igro prične igrati igralec $A$. Na začetku igre je pred igralcema
  kup $n$ vžigalic. Igralec na potezi s kupa odstrani najmanj $3$ vžigalice 
  in največ $10$ vžigalic. Izgubi tisti igralec, ki ne more izvesti poteze.

  \podnaloga
    Sestavite funkcijo \texttt{kup}, ki za dano število $n$ ugotovi,
    kateri od igralcev ima zmagovalno strategijo. Na primer:
    \begin{verbatim}
    >>> kup(2)
    B
    >>> kup(3)
    A
    >>> kup(11)
    B
    \end{verbatim}

  \podnaloga 
    Igro razširimo tako, da igralca igrata z večimi kupi vžigalic. Igralec
    na potezi si izbere kup in z njega odstrani najmanj $3$ in največ $10$
    vžigalic. Izgubi tisti igralec, ki ne more izvesti poteze. 
    Sestavite funkcijo \texttt{kupi}, ki za dan seznam naravnih števil
    ugotovi, kateri od igralcev ima zmagovalno strategijo. Na primer:
    \begin{verbatim}
    >>> kupi([11])
    B
    >>> kupi([2, 11])
    A
    >>> kupi([3, 11])
    B
    >>> kupi([2, 3, 11])
    B
    \end{verbatim}

\end{document}

